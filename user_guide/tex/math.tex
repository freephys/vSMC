\chapter{Mathematical operations}
\label{chap:Mathemtical operations}

\section{Constants}
\label{sec:Constants}

The library defines some mathematical constants in the form of
\cppinline{constexpr} functions. For example, to get the value of $\pi$ with a
desired precision, one can call the following.
\begin{cppcode}
  auto pi_f = const_pi<float>();
  auto pi_d = const_pi<double>();
  auto pi_l = const_pi<long double>();
\end{cppcode}
The compiler will evaluate these values at compile-time and thus there is no
performance difference from hard-coding the constants in the program, while the
readability is improved. All defined constants are listed in
table~\ref{tab:Mathematical constants}. Note that all functions has a prefix
\cppinline{const_}, which is omitted in the table.

\begin{table}[ht]
  \begin{tabu}{X[2l]X[l]X[2l]X[l]X[2l]X[l]}
    \toprule
    Function & Value &
    Function & Value &
    Function & Value \\
    \midrule
    \texttt{pi}             & $\pi$              &
    \texttt{pi\_2}          & $2\pi$             &
    \texttt{pi\_inv}        & $1/\pi$            \\
    \texttt{pi\_sqr}        & $\pi^2$            &
    \texttt{pi\_by2}        & $\pi/2$            &
    \texttt{pi\_by3}        & $\pi/3$            \\
    \texttt{pi\_by4}        & $\pi/4$            &
    \texttt{pi\_by6}        & $\pi/6$            &
    \texttt{pi\_2by3}       & $2\pi/3$           \\
    \texttt{pi\_3by4}       & $3\pi/4$           &
    \texttt{pi\_4by3}       & $4\pi/3$           &
    \texttt{sqrt\_pi}       & $\sqrt{\pi}$       \\
    \texttt{sqrt\_pi\_2}    & $\sqrt{2\pi}$      &
    \texttt{sqrt\_pi\_inv}  & $\sqrt{1/\pi}$     &
    \texttt{sqrt\_pi\_by2}  & $\sqrt{\pi/2}$     \\
    \texttt{sqrt\_pi\_by3}  & $\sqrt{\pi/3}$     &
    \texttt{sqrt\_pi\_by4}  & $\sqrt{\pi/4}$     &
    \texttt{sqrt\_pi\_by6}  & $\sqrt{\pi/6}$     \\
    \texttt{sqrt\_pi\_2by3} & $\sqrt{2\pi/3}$    &
    \texttt{sqrt\_pi\_3by4} & $\sqrt{3\pi/4}$    &
    \texttt{sqrt\_pi\_4by3} & $\sqrt{4\pi/3}$    \\
    \texttt{ln\_pi}         & $\ln{\pi}$         &
    \texttt{ln\_pi\_2}      & $\ln{2\pi}$        &
    \texttt{ln\_pi\_inv}    & $\ln{1/\pi}$       \\
    \texttt{ln\_pi\_by2}    & $\ln{\pi/2}$       &
    \texttt{ln\_pi\_by3}    & $\ln{\pi/3}$       &
    \texttt{ln\_pi\_by4}    & $\ln{\pi/4}$       \\
    \texttt{ln\_pi\_by6}    & $\ln{\pi/6}$       &
    \texttt{ln\_pi\_2by3}   & $\ln{2\pi/3}$      &
    \texttt{ln\_pi\_3by4}   & $\ln{3\pi/4}$      \\
    \texttt{ln\_pi\_4by3}   & $\ln{4\pi/3}$      &
    \texttt{e}              & $\EE$              &
    \texttt{e\_inv}         & $1/\EE$            \\
    \texttt{sqrt\_e}        & $\sqrt{\EE}$       &
    \texttt{sqrt\_e\_inv}   & $\sqrt{1/\EE}$     &
    \texttt{sqrt\_2}        & $\sqrt{2}$         \\
    \texttt{sqrt\_3}        & $\sqrt{3}$         &
    \texttt{sqrt\_5}        & $\sqrt{5}$         &
    \texttt{sqrt\_10}       & $\sqrt{10}$        \\
    \texttt{sqrt\_1by2}     & $\sqrt{1/2}$       &
    \texttt{sqrt\_1by3}     & $\sqrt{1/3}$       &
    \texttt{sqrt\_1by5}     & $\sqrt{1/5}$       \\
    \texttt{sqrt\_1by10}    & $\sqrt{1/10}$      &
    \texttt{ln\_2}          & $\ln{2}$           &
    \texttt{ln\_3}          & $\ln{3}$           \\
    \texttt{ln\_5}          & $\ln{5}$           &
    \texttt{ln\_10}         & $\ln{10}$          &
    \texttt{ln\_inv\_2}     & $1/\ln{2}$         \\
    \texttt{ln\_inv\_3}     & $1/\ln{3}$         &
    \texttt{ln\_inv\_5}     & $1/\ln{5}$         &
    \texttt{ln\_inv\_10}    & $1/\ln{10}$        \\
    \texttt{ln\_ln\_2}      & $\ln\ln{2}$        &
    &                    &
    &                    \\
    \bottomrule
  \end{tabu}
  \caption{Mathematical constants. Note: All functions are prefixed by
    \cppinline{const_}.}
  \label{tab:Mathematical constants}
\end{table}

\section{Vectorized operations}
\label{sec:Vectorized operations}

The library provides a set of functions for vectorized mathematical operations.
For example,
\begin{cppcode}
  std::size_t n = 1000;
  vsmc::Vector<double> a(n), b(n), y(n);
  // Fill vectors a and b
  add(n, a.data(), b.data(), y.data());
\end{cppcode}
performs addition for vectors. It is equivalent to
\begin{cppcode}
  for (std::size_t i = 0; i != n; ++i)
      y[i] = a[i] + b[i];
\end{cppcode}
The functions defined are listed in table~\ref{tab:Vectorized mathematical
  operations}.

\begin{table}[ht]
  \begin{tabu}{X[l]X[2l]X[l]X[2l]X[l]X[2l]}
    \toprule
    Function & Operation &
    Function & Operation &
    Function & Operation \\
    \midrule
    \texttt{add}     & $a + b$                               &
    \texttt{sub}     & $a - b$                               &
    \texttt{sqr}     & $a^2$                                 \\
    \texttt{mul}     & $ab$                                  &
    \texttt{abs}     & $|a|$                                 &
    \texttt{fma}     & $ab + c$                              \\
    \texttt{inv}     & $1 / a$                               &
    \texttt{div}     & $a / b$                               &
    \texttt{sqrt}    & $\sqrt{a}$                            \\
    \texttt{invsqrt} & $1 / \sqrt{a}$                        &
    \texttt{cbrt}    & $\sqrt[3]{a}$                         &
    \texttt{invcbrt} & $1 / \sqrt[3]{a}$                     \\
    \texttt{pow2o3}  & $a^{2/3}$                             &
    \texttt{pow3o2}  & $a^{3/2}$                             &
    \texttt{pow}     & $a^b$                                 \\
    \texttt{hypot}   & $\sqrt{a^2 + b^2}$                    &
    \texttt{exp}     & $\EE^a$                               &
    \texttt{exp2}    & $2^a$                                 \\
    \texttt{exp10}   & $10^a$                                &
    \texttt{expm1}   & $\EE^a - 1$                           &
    \texttt{log}     & $\ln(a)$                              \\
    \texttt{log2}    & $\log_2(a)$                           &
    \texttt{log10}   & $\log_{10}(a)$                        &
    \texttt{log1p}   & $\ln(a + 1)$                          \\
    \texttt{cos}     & $\cos(a)$                             &
    \texttt{sin}     & $\sin(a)$                             &
    \texttt{sincos}  & $\sin(a)$ and $\cos(a)$               \\
    \texttt{tan}     & $\tan(a)$                             &
    \texttt{acos}    & $\arccos(a)$                          &
    \texttt{asin}    & $\arcsin(a)$                          \\
    \texttt{atan}    & $\arctan(a)$                          &
    \texttt{acos}    & $\arccos(a)$                          &
    \texttt{atan2}   & $\arctan(a / b)$                      \\
    \texttt{cosh}    & $\cosh(a)$                            &
    \texttt{sinh}    & $\sinh(a)$                            &
    \texttt{tanh}    & $\tanh(a)$                            \\
    \texttt{acosh}   & $\mathrm{arc}\cosh(a)$                &
    \texttt{asinh}   & $\mathrm{arc}\sinh(a)$                &
    \texttt{atanh}   & $\mathrm{arc}\tanh(a)$                \\
    \texttt{erf}     & $\mathrm{erf}(a)$                     &
    \texttt{erfc}    & $\mathrm{erfc}(a)$                    &
    \texttt{cdfnorm} & $1 - \mathrm{erfc}(a / \sqrt{2}) / 2$ \\
    \texttt{lgamma}  & $\ln\Gamma(a)$                        &
    \texttt{tgamma}  & $\Gamma(a)$                           &
                     &                                       \\
    \bottomrule
  \end{tabu}
  \caption{Vectorized mathematical operations}
  \label{tab:Vectorized mathematical operations}
\end{table}

For each function, the first parameter is always the length of
the vector, and the last is a pointer to the output vector (except
\cppinline{sincos} which has two output parameters). For all functions, the
output is always a vector. If there are more than one input parameters, then
some of them, but not all, can be scalars. The order of the input parameters
are as they appear in the mathematical expressions. For example,
\cppinline{fma} perform the operation $ab + c$, then the function shall be
called as,
\begin{cppcode}
  fma(n, a, b, c, y);
\end{cppcode}
where \cppinline{a}, \cppinline{b}, \cppinline{c} are input parameters, and
some of them, not all, can be scalars instead of pointers. And \cppinline{y} is
the output parameter, which has to be pointer to a length $n$ vector.
