\chapter{Advanced usage}
\label{chap:Advanced usage}

\section{Cloning objects}
\label{sec:Cloning objects}

The \verb|Sampler<T>| and \verb|Particle<T>| objects have copy constructors,
assignment operators, move constructors, and move assignment operators that
behaves exactly the way as \cpp programmers would expect. However, these
behaviors are not always desired. For example, in \textcite{stpf} a stable
particle filter in high-dimensions was developed. Without going into the
details, the algorithm consists of a particle system where each particle is
itself a particle filter. And thus when resampling the global system, the
\verb|Sampler<T>| object will be copied, together with all of its sub-objects.
This include the \rng system within the \verb|Particle<T>| object. Even if the
user does not use this \rng system for random number generating within user
defined operations, one of these \rng will be used for resampling by the
\verb|Particle<T>| object. Direct copying the \verb|Sampler<T>| object will
lead to multiple local filters start to generating exactly the same random
numbers in the next iteration. This is an undesired side effects. In this
situation, one can clone the sampler with the following method,
\begin{cppcode}
  auto new_sampler = sampler.clone(new_rng);
\end{cppcode}
where \verb|new_rng| is a boolean value. If it is \verb|true|, then an exact
copy of \verb|sampler| will be returned, except it will have the \rng system
re-seeded. If it is \verb|false|, then the above assignemnt behaves exactly the
same as
\begin{cppcode}
  auto new_sampler = sampler;
\end{cppcode}
Alternatively, the contents of an existing \verb|Sampler<T>| object can be
replaced from another one by the following method,
\begin{cppcode}
  sampler.clone(other_sampler, retain_rng);
\end{cppcode}
where \verb|retain_rng| is a boolean value. If it is \verb|true|, then the \rng
system of \verb|other_sampler| is not copied and its own \rng system is
retained. If it is \verb|false|, then the above call behaves exactly the same
as
\begin{cppcode}
  sampler = other_sampler;
\end{cppcode}
The above method also supports move semantics. Similar \verb|clone| methods
exist for the \verb|Particle<T>| class.

\section{Customizing member types}
\label{sec:Customizing member types}

The \verb|Particle<T>| class has a few member types that can be replaced by the
user. If the class \verb|T| has the corresponding types, then the member type
of \verb|Particle<T>| will be replaced. For example, given the following
declarations inside class \verb|T|,
\begin{cppcode}
  class T
  {
      public:
      using size_type = int;
      using weight_type = /* User defined type */;
      using rng_set_type = RNGSetTBB<AES256_4x32>;
  };
\end{cppcode}
The corresponding \verb|Particle<T>::size_type|, etc., will have their defaults
replaced with the above types.

A note on \verb|weight_type|, it needs to provide the following method,
\begin{cppcode*}{texcomments}
  w.ess();           // Get $\text{\normalfont\textsc{ess}}$
  w.set_equal();     // Set $W^{(i)} = 1/N$
  w.resample_size(); // Get the size $N$.
  w.resample_data(); // Get a pointer to normalized weights
\end{cppcode*}
For the library's default class \verb|Weight|, the last two calls are the same
as \verb|w.size()| and \verb|w.data()|. However, this does not need to be so.
For example, below is the outline of an implementation of \verb|weight_type|
for distributed systems, assuming each computing node has been allocated $N_r$
particles.
\begin{cppcode*}{texcomments}
  class WeightMPI
  {
      public:
      double ess()
      {
          double local = /* $\sum_{i=1}^{N_r}(W^{(i)})^2$ */;
          double global = /* Gather local from each all nodes */;
          // Broadcast the value of global

          return 1 / global;
      }

      std::size_t resample_size() { return /* $\sum N_r$ */; }

      double *resample_data()
      {
          if (rank == 0) {
              // Gather all normalized weights into a member data on this node
              // Say resample\_weight\_
              return resample_weight_.data();
          } else {
              return nullptr;
          }
      }

      void set_equal()
      {
          // Set all weights to $1 / \sum N_r$
          // Synchronization
      }

      void set(const double *v)
      {
          // Set $W^{(i)} = v_i$ for $i = 1,\dots,N_r$
          // Compute $S_r = \sum_{i=1}^{N_r} W^{(i)}$
          // Gathering $S_r$, compute $S = \sum S_r$
          // Broadcast $S$
          // Set $W^{(i)} = W^{(i)} / S$ for $i = 1,\dots,N_r$
      }
  };
\end{cppcode*}
When \verb|Particle<T>| performs resampling, it checks if the pointer returned
by \verb|w.resample_data()| is a null pointer. It will only generate the vector
$\{a_i\}_{i=1}^N$ (see section~\ref{sub:State}) when it is not a null pointer.
And then a pointer to this vector is passed to \verb|T::copy|. Of course, the
class \verb|T| also needs to provide a suitable method \verb|copy| that can
handle the distributed system. By defining suitable \verb|WeightMPI| and
\verb|T::copy|, the library can be extended to handle distributed systems.

\section{Extending \protect\spt}
\label{sec:Extending SP}

The \verb|SingleParticle<T>| can also be extended by the user. We have already
see in section~\ref{sub:State} that if class \verb|T| is a derived class of
\verb|StateMatrix|, \verb|SingleParticle<T>| can have additional methods to
access the state. This class can be extended by defining a member class
template inside class \verb|T|. For example, for the simple particle filter in
section~\ref{sec:A simple particle filter}, we can redefine the \verb|PFState|
as the following,
\begin{cppcode}
  using PFStateBase = StateMatrix<RowMajor, 4, double>;

  template <typename T>
  using PFStateSPBase = PFStateBase::single_particle_type<T>;

  class PFState : public PFStateBase
  {
      public:
      using PFStateBase::StateMatrix;

      template <typename S>
      class single_particle_type : public PFStateSPBase<S>
      {
          public:
          using PFStateSPBase<S>::single_particle_type;

          double &pos_x() { return this->state(0); }
          double &pos_y() { return this->state(1); }
          double &vel_x() { return this->state(2); }
          double &vel_y() { return this->state(3); }

          // Return $\ell(X_t^{(i)}|Y_t)$
          double log_likelihood(std::size_t t);
      };

      void read_data(const char *param);

      private:
      Vector<double> obs_x_;
      Vector<double> obs_y_;
  };
\end{cppcode}
And later, we can use these methods when implement \verb|PFInit| etc.,
\begin{cppcode}
  class PFInit : public InitializeTBB<PFState, PFInit>
  {
      public:
      void eval_param(Particle<PFState> &particle, void *param);

      void eval_pre(Particle<PFState> &particle);

      std::size_t eval_sp(SingleParticle<PFState> sp)
      {
          NormalDistribution<double> norm_pos(0, 2);
          NormalDistribution<double> norm_vel(0, 1);
          sp.pos_x() = norm_pos(sp.rng());
          sp.pos_y() = norm_pos(sp.rng());
          sp.vel_x() = norm_vel(sp.rng());
          sp.vel_y() = norm_vel(sp.rng());
          w_[sp.id()] = sp.log_likelihood(0);

          return 0;
      }

      void eval_post(Particle<PFState> &particle);

      private:
      Vector<double> w_;
  };
\end{cppcode}
It shall be noted that, it is important to keep \verb|single_particle_type|
small and copying the object efficient. The library will frequently pass
argument of \verb|SingleParticle<T>| type by value.

\subsection{Compared to custom state type}
\label{sub:Compared to custom state type}

One can also write a custom state type. For example,
\begin{cppcode}
  class PFStateSP
  {
      public:
      double &pos_x() { return pos_x_; }
      double &pos_y() { return pos_y_; }
      double &vel_x() { return vel_x_; }
      double &vel_y() { return vel_y_; }

      double log_likelihood(double obs_x, double obs_y) const;

      private:
      double pos_x_;
      double pos_y_;
      double vel_x_;
      double vel_y_;
  };
\end{cppcode}
And the \verb|PFState| class will be defined as,
\begin{cppcode}
  using PFStateBase = StateMatrix<RowMajor, 1, PFStateSP>;

  class PFState : public PFStateBase
  {
      public:
      using PFStateBase::StateMatrix;

      double log_likelihood(std::size_t t, std::size_t i) const
      {
          return this->state(i, 0).log_likelihood(obs_x_[t], obs_y_[t]);
      }

      void read_data(const char *param);

      private:
      Vector<double> obs_x_;
      Vector<double> obs_y_;
  };
\end{cppcode}
The implementation of \verb|PFInit|, etc., will be similar. Compared to
extending the \verb|SingleParticle<T>| type, this method is perhaps more
intuitive. Functionality-wise, they are almost identical. However, there are a
few advantages of extending \verb|SingleParticle<T>|. First, it allows more
compact data storage. Consider a situation where the state space is best
represented by a real and an integer. The most intuitive way might be the
following,
\begin{cppcode}
  class S
  {
      public:
      double &x() { return x_; }
      double &u() { return u_; }

      private:
      double x_;
      int u_;
  };

  class T : StateMatrix<RowMajor, 1, S>;
\end{cppcode}
However, the type \verb|S| will need to satisfy the alignment requirement of
\verb|double|, which is 8-bytes on most platforms. However, its size might not
be a multiple of 8-bytes. Therefore the type will be padded and the storage of
a vector of such type will not be as compact as possible. This can affect
performance in some situations. An alternative approach would be the following,
\begin{cppcode}
  class T
  {
      public:
      template <typename S>
      class single_particle_type : SingleParticleBase<S>
      {
          public:
          using SingleParticleBase<S>::SingleParticleBase;

          double &x() { return this->particle().x_[this->id()]; }
          double &u() { return this->particle().u_[this->id()]; }
      };

      private:
      Vector<double> x_;
      Vector<int> u_;
  };
\end{cppcode}
By extending \verb|SingleParticle<T>|, it provides the same easy access to each
particle. However, now the state values are stored as two compact vectors.

A second advantage is that it allows easier access to the raw data. Consider
the implementation \verb|PFEval| in section~\ref{sub:Implementations}. It is
rather redundant to copy each value of the two positions, just so later we can
compute weighted sums from them. Recall that in section~\ref{sub:Monitor} we
showed that a monitor that compute the final results directly can also be added
to a sampler. Therefore, we might implement \verb|PFEval| as the following,
\begin{cppcode*}{texcomments}
  class PFEval
  {
      public:
      void operator()(std::size_t t, std::size_t dim,
          Particle<PFState> &particle, double *r)
      {
          cblas_dgemv(CblasRowMajor, CblasTrans, particle.size(), dim, 1,
              particle.value().data(), particle.value().dim(),
              particle.weight().data(), 1, 0, r, 1);
      }
  };
\end{cppcode*}
And it can be added to a sampler as,
\begin{cppcode}
  sampler.monitor("pos", 2, PFEval(), true);
\end{cppcode}
For this particular case, the performance benefit is small. But the possibility
of accessing compact vector as raw data allows easier interfacing with external
numerical libraries. If we implemented \verb|PFState| with the alternative
approach shown earlier, the above direct invoking of \verb|cblas_dgemv| will
not be possible.
