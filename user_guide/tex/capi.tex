\chapter{Interfacing with C and other languages}
\label{app:chap:Interfacing with C and other languages}

The library provides a set of C functions, declared in the header
\verb|vsmc.h|, and a companion runtime library. These functions expose a subset
of the functionality of the \cpp template library. The main purpose is for
easier interfacing with other languages, instead of implementing algorithms in
C itself. Of course, the later is possible and two complete \cnn
implementations of the simple particle filter in section~\ref{sec:A simple
  particle filter} are shown in appendices~\appref{app:sub:Sequential
  implementation in C} and~\appref{app:sub:Parallelized implementation using
  TBB in C}. See the reference manual for complete list of the \api. In this
appendix, we introduce the conventions used in the \api.

\section{Argument types}
\label{app:sec:Argument types}

For all C interfaces, \verb|int| is always used for integer types, regardless
of the \cpp interfaces, except for memory functions. And \verb|double| is
always used for floating points. For example, the following method of
\verb|Seed|,
\begin{Verbatim}
  unsigned get();
\end{Verbatim}
is exposed to C as
\begin{Verbatim}
  int vsmc_seed_get(void);
\end{Verbatim}
And \rng methods where \verb|unsigned| was expected in \cpp, an \verb|int| is
expected in C. If a function is a template, then only the specialization for
\verb|double| is exposed to C. In some cases, specializations for \verb|int|
and \verb|unsigned char| are also exposed. The use of \verb|int| as the
universal integer type is for compatibility consideration. Again, the main
purpose is for interfacing with other languages instead of implementing
algorithms in C itself. For example, Fortran does not have native unsigned
integer types. This is the main limitation of the C \api. If integer values,
such as sample size, larger than the largest value can be represented by
\verb|int| is needed. One can write their own C wrapper. Note that, when an
unsigned integer type, such as \verb|std::size_t| is expected in the \cpp
interface, and the input for the C function was a negative value, it is
silently converted to a positive value through \verb|static_cast|. Therefore
\verb|-1| will be treated as a very big number instead of emit an error. The
user is responsible to check input argument ranges. This limitation is not so
significant in practice. The types \verb|int| and \verb|double| are as portable
as one would get for interfacing with other languages. For instance, R's core
storage modes for atomic numeric vectors are either \verb|int| or
\verb|double|. Similarly, wherever an iteration is expected by the \cpp
interface, pointers to \verb|int| and \verb|double| will expected by the C
interface, with possible \verb|const| qualifiers. Enumerators are also defined
with the same values as in \cpp, but with a \verb|vSMC| prefix. For example,
\verb|Multinomial| becomes \verb|vSMCMultinomial| and the type
\verb|ResampleScheme| becomes \verb|vSMCResampleScheme|. The other two
enumerator types are \verb|vSMCMatrixLayout| and \verb|vSMCMonitorStage|.
Last, when a function object is expected in the \cpp interface, a function
pointer will be expected in the C interface. For example,
\verb|Sampler::move_type| is translated to,
\begin{Verbatim}
  typedef int (*vsmc_sampler_move_type)(int, vsmc_particle);
\end{Verbatim}

\section{Classes and methods}
\label{app:sec:Classes and methods}

A few core classes are accessible from this \api. They are listed in
table~\appref{app:tab:capi types}. Table~\appref{app:tab:capi} lists additional
features of the library exposed to the C \api apart from the core classes
listed above. With the exception of
\begin{Verbatim}
  typedef {
      double *state;
      int id;
  } vsmc_single_particle;
\end{Verbatim}
all the C types in the table are only wrappers around a pointer. For example,
\begin{Verbatim}
  typedef {
      void *ptr;
  } vsmc_sampler;
\end{Verbatim}
The pointer points to the address of a \cpp object of the corresponding type.

Each C type objects can be created by an allocation function and destroyed by a
deallocation function. For example,
\begin{Verbatim}
  vsmc_sampler sampler = vsmc_sampler_new(n, dim, vSMCMultinomial, 0.5);
\end{Verbatim}
corresponds to the \cpp calls,
\begin{Verbatim}
  Sampler<T> sampler(n, Multinomial, 0.5);
  sampler.particle.value().resize_dim(dim);
\end{Verbatim}
The memory can later be freed by
\begin{Verbatim}
  vsmc_sampler_delete(sampler);
  sampler.ptr = NULL;
\end{Verbatim}
Most non-static methods can be accesses through C. The corresponding functions
names takes the form
\texttt{\textcolor{MRed}{\textit{class}}\_\textcolor{MRed}{\textit{method}}},
where \texttt{\textcolor{MRed}{\textit{class}}} is a type name, for example
\verb|vsmc_sampler|; and \texttt{\textcolor{MRed}{\textit{method}}} is the name
of the method. The first argument will be a
\texttt{\textcolor{MRed}{\textit{class}}} type object, and the following
arguments are the same as in \cpp. When there are multiple overloading of a
method, the most general form is provided. For example,
\begin{Verbatim}
  StateMatrix<RowMajor, Dynamic, double> s;
  s.resize(n);
  s.resize(n, dim);
\end{Verbatim}
are two overloading of the \verb|resize| method. In C, only
\begin{Verbatim}
  vsmc_state_matrix s = vsmc_state_matrix_new(n, dim);
  vsmc_state_matrix_resize(s, n, dim);
\end{Verbatim}
are accessible. In a few cases, when a getter and a setter are overloaded,
\verb|get| and \verb|set| are inserted into the method name. For example,
\begin{Verbatim}
  double thres = sampler.threshold();
  sampler.threshold(thres);
\end{Verbatim}
are translated to C as
\begin{Verbatim}
  double thres = vsmc_sampler_get_threshold(sampler);
  vsmc_sampler_set_threshold(sampler, thres);
\end{Verbatim}

\begin{table}
  \begin{tabularx}{\textwidth}{lX}
    \toprule
    \cpp types & C types \\
    \midrule
    \verb|StateMatrix<RowMajor, Dynamic, double>| & \verb|vsmc_state_matrix| \\
    \verb|Weight|            & \verb|vsmc_weight|          \\
    \verb|Particle<T>|       & \verb|vsmc_particle|        \\
    \verb|SingleParticle<T>| & \verb|vsmc_single_particle| \\
    \verb|Sampler<T>|        & \verb|vsmc_sampler|         \\
    \verb|Monitor<T>|        & \verb|vsmc_monitor|         \\
    \verb|RNG|               & \verb|vsmc_rng|             \\
    \bottomrule
  \end{tabularx}
  \caption{C types . Note that, all template parameter \texttt{T} are of type
    \texttt{StateMatrix<RowMajor, Dynamic, double>}}
  \label{app:tab:capi types}
\end{table}

\begin{table}
  \begin{tabularx}{\textwidth}{lX}
    \toprule
    Feature & Notes \\
    \midrule
    \smp backends
    & Section~\ref{sec:Symmetric multiprocessing} \\
    Resamling algorithms
    & Section~\ref{sec:Builtin algorithms} \\
    Ordered uniform random numbers
    & Section~\ref{sec:User defined algorithms} \\
    Seeding
    & Section~\ref{sub:Seeding counter-based RNG} \\
    Vectorized random number generating
    & Section~\ref{sec:Distributions} \\
    Aligned memory allocation
    & Section~\ref{sec:Aligned memory allocation} \\
    Covariance matrix
    & Section~\ref{sec:Sample covariance} \\
    Process program command line options
    & Section~\ref{sec:Process command line program options} \\
    Display program progress
    & Section~\ref{sec:Display program progress} \\
    Stop watch
    & Section~\ref{sec:Stop watch} \\
    Registering \cppoo{} \rng as \mkl{} \brng
    & Reference manual \\
    Random walk
    & Reference manual \\
    \bottomrule
  \end{tabularx}
  \caption{Features accessible from C}
  \label{app:tab:capi}
\end{table}
