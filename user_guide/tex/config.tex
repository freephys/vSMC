\chapter{Configuration macros}
\label{chap:Configuration macros}

The library has a few configuration macros. All these macros can be overwritten
by the user by defining them with proper values before including any of the
library's headers. All configurations macros are listed in
table~\ref{tab:Configuration macros}. There are some additional macros for \rng
related functionalities. They will be discussed in chapter~\ref{chap:Random
  number generating}.

\begin{table}
  \begin{tabularx}{\textwidth}{llX}
    \toprule
    Macro & Default & Description \\
    \midrule
    \verb|VSMC_HAS_INT128| & Platform dependent
    & Support for 128-bits integers \\
    \verb|VSMC_HAS_SSE2| & Platform dependent
    & Support for \sse{}2 intrinsic functions \\
    \verb|VSMC_HAS_AVX2| & Platform dependent
    & Support for \avx{}2 intrinsic functions \\
    \verb|VSMC_HAS_AES_NI| & Platform dependent
    & Support for \aesni intrinsic functions \\
    \verb|VSMC_HAS_RDRAND| & Platform dependent
    & Support for \rdrand intrinsic functions \\
    \verb|VSMC_HAS_X86| & Platform dependent
    & Support for x86 platform \\
    \verb|VSMC_HAS_X86_64| & Platform dependent
    & Support for x86-64 platform \\
    \verb|VSMC_HAS_POSIX|  & Platform dependent
    & Support for \posix platform \\
    \verb|VSMC_HAS_OMP| & Platform dependent
    & Support for OpenMP~3.0 or higher \\
    \verb|VSMC_HAS_TBB| & \verb|0|
    & Support for \tbb~4.0 or higher \\
    \verb|VSMC_HAS_TBB_MALLOC| & \verb|VSMC_HAS_TBB|
    & Support for \tbb scalable memory allocation \\
    \verb|VSMC_HAS_HDF5| & \verb|0| &
    Support for \hdf~1.8.6 or higher \\
    \verb|VSMC_HAS_MKL| & \verb|0|
    & Support for \mkl~11.0 or higher \\
    \midrule
    \verb|VSMC_USE_OMP| & \verb|VSMC_HAS_OMP|
    & Use OpenMP parallelization outside the \smp module \\
    \verb|VSMC_USE_TBB| & \verb|VSMC_HAS_TBB|
    & Use \tbb parallelization outside the \smp module \\
    \verb|VSMC_USE_MKL_CBLAS| & \verb|VSMC_HAS_MKL|
    & Use \verb|mkl_cblas.h| instead of \verb|cblas.h| \\
    \verb|VSMC_USE_MKL_LAPACKE| & \verb|VSMC_HAS_MKL|
    & Use \verb|mkl_lapacke.h| instead of \verb|lapacke.h| \\
    \verb|VSMC_USE_MKL_VML| & \verb|VSMC_HAS_MKL|
    & Use \mkl vector mathematical functions (\vml) \\
    \verb|VSMC_USE_MKL_VSL| & \verb|VSMC_HAS_MKL|
    & Use \mkl statistical functions (\vsl) \\
    \verb|VSMC_USE_ACCELERATE| & Platform dependent
    & Use Mac OS X Accelerate framework for \blas. Ignored if
    \verb|VSMC_USE_MKL_BLAS| is defined to a non-zero value. \\
    \midrule
    \verb|VSMC_INT64| & Platform dependent
    & The 64-bit integer type used by x86 intrinsics \\
    \verb|VSMC_INT128| & Platform dependent
    & The 128-bit integer type \\
    \verb|VSMC_CBLAS_INT_TYPE| & \verb|int|
    & The default integer type of \blas routines \\
    \verb|VSMC_ALIGNMENT| & \verb|32|
    & Default alignment for scalar types \\
    \verb|VSMC_ALIGNMENT_MIN| & \verb|16|
    & Minimum alignment for all types \\
    \verb|VSMC_ALIGNED_MEMORY_TYPE| & Platform dependent
    & The type of \verb|AlignedMemory| \\
    \verb|VSMC_CONSTRUCT_SCALAR| & \verb|0|
    & Should \verb|Allocator::construct| zero out scalar types \\
    \bottomrule
  \end{tabularx}
  \caption{Configuration macros}
  \label{tab:Configuration macros}
\end{table}

There are three types of configuration macros. The first type has a prefix
\verb|VSMC_HAS|. These macros specify a certain feature or third-party library
is available. The second type has a prefix \verb|VSMC_USE|. These macros
specify that a certain feature or third-party library shall be used, if
available. For example, if \verb|VSMC_HAS_MKL| is defined to a non-zero value,
but the it is desirable not to use \mkl's vector math functions, then one can
define \verb|VSMC_USE_MKL_VML| to zero to prevent the library to use this
individual component. All other macros define either types or constants that
are used by the library.

Another important difference between macros with prefixes \verb|VSMC_HAS| and
\verb|VSMC_USE| is that, the former will affect the interface while the later
only affect internal implementations.
