\chapter{Utilities}
\label{chap:Utilities}

The library provides some utilities for writing Monte Carlo simulation
programs. For some of them, such as command line option processing, there are
more advanced, dedicated libraries out there. The library only provides some
basic functionality that is sufficient for most simple cases.

\section{Aligned memory allocation}
\label{sec:Aligned memory allocation}

The standard library class \cppinline{std::allocator} is used by containers to
allocate memory. It works fine in most cases. However, sometime it is desired
to allocate memory aligned by a certain boundary. The library provides the
class template,
\begin{cppcode}
  template <typename T, std::size_t Alignment = VSMC_ALIGNMENT,
      typename Memory = AlignedMemory>
  class AlignedAllocator;
\end{cppcode}
where the configuration macro \cppinline{VSMC_ALIGNMENT} is defined to be
\cppinline{32} by default. For the requirement of the parameter type
\cppinline{Memory}, see the reference manual. It is sufficient to mention here
that the default implementation works best if \tbb is available. This class can
be used as a drop-in replacement of \cppinline{std::allocator<T>}. In fact,
this library defines a type alias \cppinline{Vector<T>} which is
\cppinline{std::vector<T, AlignedAllocator<T>>} if \cppinline{T} is a scalar
type, and \cppinline{std::vector<T>} otherwise.

\section{Sample covariance estimating}
\label{sec:Sample covariance estimating}

The library provides some basic functionality to estimate sample variance. For
example,
\begin{cppcode}
  constexpr std::size_t d = /* Dimension */;
  using T = StateMatrix<RowMajor, d, double>;
  Sampler<T> sampler(N);
  // operations on the sampler
  double mean[d];
  double cov[d * d];
  Covariance eval;
  auto x = sampler.particle().value().data();
  auto w = sampler.particle().weight().data();
  eval(RowMajor, N, d, x, w, mean, cov);
\end{cppcode}
The sample covariance matrix will be computed and stored in \cppinline{cov}.
The mean vector is stored in \cppinline{mean}. Note that, if any of them is a
null pointer, then the corresponding output is not computed. The sample
\cppinline{x} is assumed to be stored in an $N$ by $d$ matrix. The first
argument passed to \cppinline{eval} is the storage layout of this matrix. If
\cppinline{x} is a null pointer, then no computation will be done. If
\cppinline{w} is a null pointer, then the weight is assumed to be equal for all
samples. The method has three additional optional parameters. The first is
\cppinline{cov_layout}, which specifies the covariance matrix storage layout.
The second is \cppinline{cov_upper} and the third \cppinline{cov_packed}, both
are \cppinline{false} by default. If the later is \cppinline{cov_packed}, a
packed vector of length $d(d+1)/2$ is written into \cppinline{cov}. If
\cppinline{cov_upper} is \cppinline{false}, then the upper triangular is
packed, otherwise the lower triangular is packed.

The estimated covariance matrix is often used to construct multivariate Normal
distribution for the purpose of generating random walk scales. The
\cppinline{NormalMVDistribution} in section~\ref{sec:Distributions} accepts
the lower triangular of the Cholesky decomposition of the covariance instead of
the covariance matrix itself. The following function will compute this
decomposition,
\begin{cppcode}
  double chol[d * (d + 1) / 2];
  cov_chol(d, cov, chol);
\end{cppcode}
The output \cppinline{chol} is a packed vector in row major storage. This
function also has three optional parameters, which are the same as those of
\cppinline{Covariance::operator()}, except that they are now used to specify
the storage scheme of the input parameter \cppinline{cov}.

\section{Storing objects in \protect\hdf}
\label{sec:Storing objects in HDF5}

If the \hdf library is available, it is possible to store
\cppinline{Sampler<T>} objects, etc., in the \hdf format. For example,
\begin{cppcode}
  hdf5store(sampler, "pf.h5", "sampler");
\end{cppcode}
create a \hdf file with the sampler stored as a list. In R it can be processed
as the following,
\begin{rcode}
  library(rhdf5)
  pf <- as.data.frame(h5read("pf.h5", "sampler"))
\end{rcode}
This creates a \rinline{data.frame} similar to that shown in
section~\ref{sub:Implementations}. Other types of objects can also be store,
see the reference manual for details.

\section{\protect\raii classes for \protect\mkl pointers}
\label{sec:RAII classes for MKL pointers}

The library provides a few classes to manage \mkl pointers. It provides
Resource Acquisition Is Initialization (\raii) idiom on top of the \mkl C
interface. For example,
\begin{cppcode}
  // VSLSSTaskPtr ptr;
  // vsldSSNewTask(&ptr, &p, &n, &xstorage, x, w, indices);
  MKLSSTask<double> task(&p, &n, &xstorage, x, w, indices);
  // vsldSSEditMoments(ptr, mean, r2m, r3m, r4m, c2m, c3m, c4m);
  task.edit_moments(mean, r2m, r3m, r4m, c2m, c3m, c4m);
  // vsldSSCompute(ptr, estimates, method);
  task.compute(estimates, method)
  // vslSSDeleteTask(&ptr);
\end{cppcode}
In the above snippets, \cppinline{MKLSSTask} manages a \cppinline{VSLSSTaskPtr}
task pointer. All C functions that operates on the pointer, is also defined as
methods in the class. Table~\ref{tab:RAII classes for MKL pointers} lists the
classes defined by the library and their corresponding \mkl pointers.

\begin{table}[t]
  \begin{tabu}{X[l]X[l]}
    \toprule
    Class & \mkl pointer type \\
    \midrule
    \texttt{MKLStream}   & \texttt{VSLStreamStatePtr} \\
    \texttt{MKLSSTask}   & \texttt{VSLSSTaskPtr}      \\
    \texttt{MKLConvTask} & \texttt{VSLConvTask}       \\
    \texttt{MKLCorrTask} & \texttt{VSLCorrTask}       \\
    \texttt{MKLDFTask}   & \texttt{DFTaskPtr}         \\
    \bottomrule
  \end{tabu}
  \caption{\protect\raii classes for \protect\mkl pointers}
  \label{tab:RAII classes for MKL pointers}
\end{table}

\section{Program options}
\label{sec:Program options}

The library provides some basic support of processing command line program
options. Here we show a minimal example. The complete program is shown in
appendix~\appref{sec:Processing command line program options}. First, one need to
allocated variables to store the options to be processed.
\begin{cppcode}
  int n;
  std::string str;
  std::vector<double> vec;
\end{cppcode}
All types that support standard library \io stream operations are supported. In
addition, \cppinline{std::vector<T>}, where \cppinline{T} is a type that
supports standard library \io stream operations is also supported. Then,
\begin{cppcode}
  ProgramOptionMap option_map;
\end{cppcode}
constructs the container of options. Options can be added to the map,
\begin{cppcode}
  option_map
      .add("str", "A string option with a default value", &str, "default")
      .add("n", "An integer option", &n)
      .add("vec", "A vector option", &vec);
\end{cppcode}
The first argument is the name of the option, the second is a description, and
the third is a pointer to where the option's value shall be stored. The last
optional argument is a default value. The options on the command line can be
processed as the following,
\begin{cppcode}
  option_map.process(argc, argv);
\end{cppcode}
where \cppinline{argc} and \cppinline{argv} are the arguments of the
\cppinline{main} function. When the program is invoked, each option can be
passed like below,
\begin{textcode}
  ./program_option --vec 1 2 1e-1 --str "abc" --vec 8 9 --str "def hij" --n 2 4
\end{textcode}
The results of the option processing is displayed below,
\begin{textcode}
  n: 4
  str: def hij
  vec: 1 2 0.1 8 9
\end{textcode}
To summarize these output, the same option can be specified multiple times. If
it is a scalar option, the last one is used (\textinline{--str},
\textinline{--n}). A string option's value can be grouped by quotes. For a
vector option (\textinline{--vec}), all values are gather together and inserted
into the vector.

\section{Program progress}
\label{sec:Program progress}

Sometime it is desirable to see how much progress of a program has been made.
The library provide a \cppinline{Progress} class for this purpose. Here we show
a minimal example. The complete program is shown in
appendix~\appref{sec:Display program progress}.
\begin{cppcode}
  vsmc::Progress progress;
  progress.start(n * n);
  for (std::size_t i = 0; i != n; ++i) {
      std::stringstream ss;
      ss << "i = " << i;
      progress.message(ss.str());
      for (std::size_t j = 0; j != n; ++j) {
          // Do some computation
          progress.increment();
      }
  }
  progress.stop();
\end{cppcode}
When invoked, the program output something similar the below
\begin{textcode}
  [  4%][00:07][  49019/1000000][i = 49]
\end{textcode}
The method \cppinline{progress.start(n * n)} starts the printing of the
progress. The argument specifies how many iterations there will be before it is
stopped. The method \cppinline{progress.message(ss.str())} direct the program
to print a message. This is optional. Each time after we finish $n$ iterations,
we increment the progress count by calling \cppinline{progress.increment()}.
And after everything is finished, the method \cppinline{progress.stop()} is
called.

\section{Timing}
\label{sec:Timing}

Performance can only be improved after it is first properly benchmarked. There
are advanced profiling programs for this purpose. However, sometime simple
timing facilities are enough. The library provides a simple class
\cppinline{StopWatch} for this purpose. As its name suggests, it works much
like a physical stop watch. Here is a simple example
\begin{cppcode}
  StopWatch watch;
  for (std::size_t i = 0; i != n; ++i) {
      // Some computation
      watch.start();
      // Computation to be benchmarked;
      watch.stop();
      // Some other computation
  }
  double t = watch.seconds(); // The time in seconds
\end{cppcode}
The above example demonstrate that timing can be accumulated between loop
iterations, function calls, etc. It shall be noted that, the time is only
accurate if the computation between \cppinline{watch.start()} and
\cppinline{watch.stop()} is non-trivial.
