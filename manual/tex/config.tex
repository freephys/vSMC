\chapter{Configuration macros}
\label{chap:Configuration macros}

The library has a few configuration macros. All macros have default values if
left undefined, and can be overwritten by defining them with proper values
before including any of the library headers.

There are three types of configuration macros. The first type has a prefix
\verb|VSMC_HAS_|. These macros specify a certain feature or third-party library
is available. The second type has a prefix \verb|VSMC_USE_|. These macros
specify that a certain feature or third-party library can be used, if
available. These two kinds of macros shall only be defined to integral values,
with zero as false and all other values as true. The third type can be defined
to arbitrary contents, and usually affect implementation choices made by the
library.

A concrete example of the relations among these three types of configuration
macros is the memory allocation functions used by the library. The details is
in section~\ref{sec:Aligned memory allocation}. In short, they library has
three implementations for aligned memory allocation. The preferred method is to
use the \verb|scalable_aligned_malloc| from \tbb. It is defined through a class
called \verb|AlignedMemoryTBB|. This class is only defined if and only if
\verb|VSMC_HAS_TBB| is true, in which case the runtime library \verb|tbbmalloc|
will need to be linked as well (\verb|-ltbbmalloc| on \unix-alike systems).
However, if it is desirable to define this class such that the user can use it
with \stl containers, but it is undesirable for the library itself to use it
everywhere, then one can further define the macro \verb|VSMC_USE_TBB_MALLOC| to
zero. In this case, the class is still defined, but it will not be used by the
library. Instead the system will now prefer operating system dependent memory
allocation functions such as \verb|posix_memalign|. Now, suppose this is still
undesirable, and one want to use a memory allocation function not directly
supported by the library. In this case, one can implement a aligned memory
class (see section~\ref{sec:Aligned memory allocation} for details), and then
define the macro \verb|VSMC_ALIGNED_MEOMRY_TYPE| to this class. For example,
\begin{Verbatim}
  class AlignedMemoryUsr
  {
      public:
      static void *aligned_malloc(std::size_t n, std::size_t alignment);
      static void aligned_free(void *ptr);
  };

  #define VSMC_ALIGNED_MEMORY_TYPE ::AlignedMemoryUsr
  // Include vSMC headers
\end{Verbatim}

In summary, the \verb|VSMC_HAS_| macros affect what will be \emph{defined} by
the library. The \verb|VSMC_USE_| macros affect what will be \emph{used} by the
library. And other macros give direct control of the library's implementations.
In most situations, it is sufficient to use the first two to configure the
features of the library. The third kind is for more advanced users. Below we
details all configuration macros.

\section{Platform characteristics}
\label{sec:Platform characteristics}

The following macros have default values that depend on the compiler, operating
system or \cpu types. Their default values are therefore platform dependent.
The library tries its best to determine the correct values. But in the case of
errors, define these macros to override the default values.

\paragraph{\texttt{VSMC\_HAS\_X86}} Determine if the \cpu is an x86 compatible.
In few rare occasions the library will use features known to exist on x86
\cpu{}s, for example the little-endian memory layout, etc.

\paragraph{\texttt{VSMC\_HAS\_X86\_64}} Determine if the \cpu is x86-64
compatible.

\paragraph{\texttt{VSMC\_HAS\_POSIX}} Determine if the operating system is
\posix compatible.

\paragraph{\texttt{VSMC\_HAS\_AES\_NI}} Determine if the \aesni instructions
are supported by the \cpu and the compiler. The \rng{}s based on these
instructions in section~\ref{sub:AES-NI instructions based RNG} will are
defined if this macro is true. One can include the following header directly to
bypass this check,
\begin{Verbatim}
  #include <vsmc/rng/aes_ni.hpp>
\end{Verbatim}

\paragraph{\texttt{VSMC\_HAS\_RDRAND}} Determine if the \rdrand instructions
are supported by the \cpu and the compiler. The \rng{}s based on these
instructions in section~\ref{sec:Non-deterministic RNG} are only defined if
this macro is true. One can include the following header directly to bypass
this check,
\begin{Verbatim}
  #include <vsmc/rng/rdrand.hpp>
\end{Verbatim}

\paragraph{\texttt{VSMC\_HAS\_INT128}} Determine if the compiler support
128-bits integer types.

\paragraph{\texttt{VSMC\_INTE128}} The type of the 128-bits integer. Only have
effects when \verb|VSMC_HAS_INT128| is true. The default is \verb|__int128|.

\section{Third-party libraries}
\label{sec:Third-party libraries}

The following macros determine if certain third-party libraries are available.
All macros with the prefix \verb|VSMC_HAS_| are define to \verb|0| by default.
All macros with the prefix \verb|VSMC_USE_| are defined to \verb|1| if the
corresponding \verb|VSMC_HAS_| macro is true. Otherwise it is defined to
\verb|0|, unless stated otherwise.

\paragraph{\texttt{VSMC\_HAS\_OMP}} Determine if OpenMP is supported.

\paragraph{\texttt{VSMC\_USE\_OMP}} Allow the library to use OpenMP for
parallelization. Currently this macro has no affect. The use of OpenMP for
parallelization is explicitly through \verb|MoveOMP| etc.

\paragraph{\texttt{VSMC\_HAS\_OPENCL}} Determine if OpenCL (version~1.2 or
higher) is supported.

\paragraph{\texttt{VSMC\_HAS\_HDF5}} Determine if the \hdf library is
available. The functions in section~\ref{sec:Store objects in HDF5 format} are
only defined if this macro is true. One can include the following header
directly to bypass this check,
\begin{Verbatim}
  #include <vsmc/utility/hdf5.hpp>
\end{Verbatim}

\paragraph{\texttt{VSMC\_HAS\_TBB}} Determine if the \tbb library is available.
The \verb|AlignedMemoryTBB| class in section~\ref{sec:Aligned memory
  allocation} and the \verb|RNGSetTBB| class template in
section~\ref{sec:Multiple RNG streams} are only defined if this macro is true.

\paragraph{\texttt{VSMC\_USE\_TBB}} Allow the library to use the \tbb library
for parallelization. Currently this macro has no affect. The use of \tbb for
parallelization is explicitly through \verb|MoveTBB| etc.

\paragraph{\texttt{VSMC\_USE\_TBB\_MALLOC}} Allow the library to use
\verb|AlignedMemoryTBB| as the default memory allocation method.

\paragraph{\texttt{VSMC\_USE\_TBB\_TLS}} Allow the library to use
\verb|RNGSetTBB| as the default multiple \rng stream management method.

\paragraph{\texttt{VSMC\_HAS\_MKL}} Determine if the \mkl library is available.
The \rng{}s in section~\ref{sec:MKL RNG} are only defined if this macro is
true. One can include the following header directly to bypass this check,
\begin{Verbatim}
  #include <vsmc/rng/mkl.hpp>
\end{Verbatim}

\paragraph{\texttt{VSMC\_USE\_MKL\_CBLAS}} Allow the library to use the
\verb|mkl_cblas.h| header and assuming that corresponding runtime library will
be linked for \blas support.

\paragraph{\texttt{VSMC\_USE\_MKL\_VML}} Allow the library to accelerate the
vector functions in section~\ref{sec:Vectorized operations} using the \vml
component of \mkl.

\paragraph{\texttt{VSMC\_USE\_MKL\_VSL}} Allow the library to use the \vsl
component of \mkl.

\paragraph{\texttt{VSMC\_USE\_ACCELERATE}} Allow the library to the Accelerate
framework on Mac OS X for \blas support. The \verb|VSMC_USE_MKL_CBLAS| will
take precedence over this macro. In addition, it also allows the library to use
this framework to accelerate some vector functions in
section~\ref{sec:Vectorized operations}. The \verb|VSMC_USE_MKL_VML| will take
precedence over this macro for this purpose.

\paragraph{\texttt{VSMC\_USE\_CBLAS}} Allow the library to the standard C
interface of \blas. The default value is \verb|1| is either
\verb|VSMC_USE_CBLAS| or \verb|VSMC_USE_ACCELERATE| is true. Otherwise the
default value is zero. Manually define this macro to true if the \verb|cblas.h|
header is available and a compatible runtime library is linked. When this macro
is tested to be false, the library will declare the \blas Fortran routines in C
itself (see below).

\paragraph{\texttt{VSMC\_BLAS\_NAME} and
  \texttt{VSMC\_BLAS\_NAME\_NO\_UNDERSCORE}} These two macros determine how
shall the \blas Fortran routines be declared in C. The default behavior is to
append an underscore to the function name. For example, \verb|dgemv| in Fortran
becomes \verb|dgemv_| in C. If there should be no underscore, define the second
macro. If the name mangling is more complicated, one can define the
\verb|VSMC_BLAS_NAME| macro directly. For example,
\begin{Verbatim}
  #define VSMC_BLAS_NAME(x) _##x
\end{Verbatim}
These macros only have effects if \verb|VSMC_USE_CBLAS| is false.

\paragraph{\texttt{VSMC\_BLAS\_INT}} The integer type of \blas routines. The
default is \verb|MKL_INT| if \verb|VSMC_USE_CBLAS| and
\verb|VSMC_USE_MKL_CBLAS| are both true. Otherwise it is \verb|int|. If the
\blas interface use another integer type, such as the \ilp{}64 interface of
some implementations on \lp{}64 platforms, then one should redefine this macro
to the correct type.

\section{RNG engines}
\label{sec:RNG engines}

Some configuration macros of counter-based \rng{}s are discussed in
section~\ref{sec:Counter-based RNG}.

\paragraph{\texttt{VSMC\_RNG\_TYPE}} The type of the alias \verb|RNG|. The
default is \verb|ARS| if \verb|VSMC_HAS_AES_NI| is true. Otherwise, it is
\verb|Threefry|.

\paragraph{\texttt{VSMC\_RNG\_64\_TYPE}} The type of the alias \verb|RNG_64|.
The default is \verb|ARS_64| if \verb|VSMC_HAS_AES_NI| is true. Otherwise, it
is \verb|Threefry_64|.

\paragraph{\texttt{VSMC\_RNG\_MINI\_TYPE}} The type of the alias
\verb|RNGMini|.  The default is \verb|Philox2x32|.

\paragraph{\texttt{VSMC\_RNG\_MINI\_64\_TYPE}} The type of the alias
\verb|RNGMini_64|. The default is \verb|Philox2x32_64|.

\paragraph{\texttt{VSMC\_RNG\_SET\_TYPE}} The type of the alias \verb|RNGSet|.
The default is \verb|RNGSetTBB| if \verb|VSMC_USE_TBB_TLS| is true. Otherwise,
it is \verb|RNGSetVector|. Note that, the two class templates have different
default template argument. The class \verb|RNGSetVector<>| is the same as
\verb|RNGSetVector<RNGMIni>| while the class \verb|RNGSetTBB<>| is the same as
\verb|RNGSetTBB<RNG>|.

\section{Memory allocation}
\label{sec:Memory allocation}

\paragraph{\texttt{VSMC\_ALIGNMENT}} The default alignment for scalar types,
such as \verb|int|. More specifically, this affects types such that
\verb|std::is_scalar<T>| true. The default is \verb|32|. This is sufficient for
modern \simd operations. This will affect the memory allocated by
\verb|Vector<T>| on the heap and \verb|Array<T>| on the stack. The value must
be a power of two and positive.

\paragraph{\texttt{VSMC\_ALIGNMENT\_MIN}} The minimum alignment for all types.
The default is \verb|16|. The value must be a power of two and positive.

\paragraph{\texttt{VSMC\_ALIGNED\_MEMORY\_TYPE}} The default type of the
\verb|Memroy| template parameter of \verb|Allocator| (see
section~\ref{sec:Aligned memory allocation}). The default is
\verb|AlignedMemroyTBB| if \verb|VSMC_USE_TBB_MALLOC| is true. Otherwise it is
\verb|AlignedMemorySYS| if \verb|VSMC_HAS_POSIX| is true or use the \msvc
compiler. Otherwise, it is \verb|AlignedMemorySTD|. To use other memory
allocation libraries, one usually does not need to define a new class. Most
such libraries provides proxies to transparently replace system allocation
functions such as \verb|posix_memalign| and \verb|_aligned_malloc| etc. And
thus it is sufficient to define this macro to \verb|AlignedMemorySYS| and link
to the proper libraries.

\paragraph{\texttt{VSMC\_CONSTRUCT\_SCALAR}} Determine if scalar types shall be
zero initialized upon allocation by \verb|Allocator|. The default value is
\verb|0|. This departure from the behavior of \verb|std::allocator|, with which
the memory is always value initialized. For example,
\begin{Verbatim}
  std::vector<int> v(n);
\end{Verbatim}
will initialize all values to zero. In contrast,
\begin{Verbatim}
  Vector<int> v(n);
\end{Verbatim}
will leave the memory uninitialized. To get zero initialized vectors, one can
either define this macro to \verb|1|, or more efficiently,
\begin{Verbatim}
  Vector<int> v(n);
  std::fill(v.begin(), v.end(), 0); // or
  std::memset(v.data(), 0, sizeof(int) * v.size());
\end{Verbatim}
Most standard library implementations will pass the \verb|std::fill| call to
\verb|std::memset|. It is strongly recommended that to leave this macro
defined to zero. The \verb|std::vector|, with which \verb|Vector| is an alias
to, calls the \verb|construct| member of \verb|Allocator| to initialize
elements one by one, which is a huge waste of time. The library does not rely
on zero initialization itself. Note that, this macro only affect
\emph{default initialization of scalars}. For class types etc., and other
constructor of \verb|std::vector|, such as
\begin{Verbatim}
  Vector<ClassType> c(n);
  Vector<int> v(n, 2);
\end{Verbatim}
the behavior will be as expected. That is, the first will be default
initialized and the second will be value initialized.

\section{Error handling}
\label{sec:Error handling}

\paragraph{\texttt{VSMC\_NO\_RUNTIME\_ASSERT}} Determine if all runtime
assertions shall be disabled. The default is \verb|1| if \verb|NDEBUG| is
defined. Otherwise, it is \verb|0|. Runtime assertions are hard errors. The
program will be terminated by \verb|std::exit| upon failure of assertions.
These are usually errors that will cause undefined behaviors.

\paragraph{\texttt{VSMC\_NO\_RUNTIME\_WARNING}} Determine if all runtime
warnings shall be disabled. The default is \verb|1| if \verb|NDEBUG| is
defined. Otherwise, it is \verb|0|. Runtime warnings are soft errors. The
program will not be terminated and will continue. These errors do not introduce
undefined behaviors, but indicate possible misuse of library features.

\paragraph{\texttt{VSMC\_RUNTIME\_ASSERT\_AS\_EXCEPTION}} Determine if all
runtime assertions shall be turned into exceptions. The exception type is
\verb|RuntimeAssert|. The default value is \verb|0|. It has no affect is
\verb|VSMC_NO_RUNTIME_ASSERT| is true.

\paragraph{\texttt{VSMC\_RUNTIME\_WARNING\_AS\_EXCEPTION}} Determine if all
runtime warnings shall be turned into exceptions. The exception type is
\verb|RuntimeWarning|. The default value is \verb|0|. It has no affect is
\verb|VSMC_NO_RUNTIME_WARNING| is true.
